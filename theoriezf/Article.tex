\documentclass[a4paper]{article}

\usepackage[top=3.5cm, bottom=3.5cm, left=3.5cm, right=3.5cm]{geometry}
\usepackage{ae,lmodern}
\usepackage[english]{babel}
\usepackage[utf8]{inputenc}
\usepackage[T1]{fontenc}

%\usepackage{paralist, tabularx}
\usepackage{oldlfont,amssymb,epsf,stmaryrd,epsfig,color}\usepackage{amsmath}
%\usepackage{hyperref}
%\usepackage{framed,multicol,changepage}
%\usepackage{graphicx}
\usepackage[dvipsnames]{xcolor}
%\usepackage{skull,ifoddpage,marginnote}
%\usepackage{pifont}
%\usepackage{graphicx}
%\usepackage{ebproof}
%\usepackage{mathtools}
%\usepackage{bm}
%\usepackage{stmaryrd}
%\usepackage{cleveref}
%\usepackage{tikz}
%\usepackage{import} % For inkscape generated files

\title{Implementation of Set theory in \dedukti}
\date{}

\def\Type{\mbox{\tt TYPE}}
\def\Kind{\mbox{\tt KIND}}
\def\ra{\rightarrow}
\def\lra{\longrightarrow}
\def\Set{\mbox{\it Set}}
\def\bliota{{\color{blue} ind}}
\def\El{{\mbox{\it El}}}
\def\Prop{{\mbox{\it Prop}}}
\def\Prf{{\color{blue} \mbox{\it Prf}}}
\def\imp{\mathbin{\color{blue} \Rightarrow}}
\def\fa{{\color{blue} \forall}}
\def\bltop{{\color{blue} \top}}
\def\blbot{{\color{blue} \bot}}
\DeclareMathOperator{\blneg}{{\color{blue} \neg}}
\def\conj{\mathbin{\color{blue} \wedge}}
\def\disj{\mathbin{\color{blue} \vee}}
\def\ex{{\color{blue} \exists}}
\def\S{{\color{blue} \mbox{\it succ}}}
\def\0{{\color{blue} 0}}
\def\positive{{\color{blue} \mbox{\it positive}}}
\def\pred{{\color{blue} \mbox{\it pred}}}
\def\Prfc{{\color{blue} \mbox{\it Prf}_c}}
\def\impc{\mathbin{\color{blue} \Rightarrow_c}}
\def\conjc{\mathbin{\color{blue} \wedge_c}}
\def\disjc{\mathbin{\color{blue} \vee_c}}
\def\fac{{\color{blue} \forall_c}}
\def\exc{{\color{blue} \exists_c}}
\def\o{{\color{blue} o}}
\def\arr{\mathbin{\color{blue} \leadsto}}%\mbox{\it arrow}}}
\def\arrd{\mathbin{\color{blue} \leadsto_d}}%\mbox{\it arrow}_d}}
\def\impd{\mathbin{\color{blue} \Rightarrow_d}}
\def\blpi{{\color{blue} \pi}}
\def\Scheme{{\color{blue} \mbox{\it Scheme}}}
\def\Els{{\color{blue} \mbox{\it Els}}}
\def\lift{{\color{blue} \uparrow}}
\def\calfa{{\rotatebox[origin=c]{180}{{\color{blue} A}}}}
\def\sffa{{\rotatebox[origin=c]{180}{{\color{blue}\ensuremath{\mathcal{A}}}}}}
\def\pos{\mbox{\it positive}}
\def\psub{{\color{blue} \mbox{\it psub}}}
\def\pair{{\color{blue} \mbox{\it pair}}}
\def\fst{{\color{blue} \mbox{\it fst}}}
\def\snd{{\color{blue} \mbox{\it snd}}}
\def\eq{{\mbox{\it eq}}}
\def\refl{{\mbox{\it refl}}}

\newtheorem{definition}{Definition}[section]
\newtheorem{theorem}{Theorem}[section]
\newtheorem{lemma}{Lemma}[section]
\newtheorem{corollary}{Corollary}[section]
\newtheorem{property}{Property}[section]

\newenvironment{proof}{\noindent {\em Proof.}}{\medskip}
\newenvironment{example}{\noindent {\em Example.}}{\medskip}
\newenvironment{remark}{\noindent {\em Remark.}}{\medskip}

\colorlet{ftcolor}{Aquamarine}

\newcommand{\mute}[1]{\ifdraft{#1} \else{} \fi}

\newcounter{WarnCounts}
\newcommand{\decorateWC}{
  \stepcounter{WarnCounts}
  \checkoddpage
  \ifoddpageoroneside
     \mbox{}\marginnote{\textcolor{red}{$\skull_{\theWarnCounts}$}}
     \else
     \reversemarginpar
     \mbox{}\marginnote{\textcolor{red}{$\skull_{\theWarnCounts}$}}
     \normalmarginpar
   \fi
}

\newcounter{ToDos}
\newcommand{\decorateTD}{
  \stepcounter{ToDos}
    \checkoddpage
    \ifoddpageoroneside
     \mbox{}\marginnote{\textcolor{red}{$\skull_{\theToDos}$}}
     \else
     \reversemarginpar
     \mbox{}\marginnote{\textcolor{red}{$\skull_{\theToDos}$}}
     \normalmarginpar
   \fi
}

% Skeleton for remark comments
% 1: Name, 2: Color, 3:Comment
\newcommand{\signedComment}[3]
           {\mute{\decorateWC\textcolor{#2}{(#1:~{#3})}}}

% remarks
\newcommand{\todo}[1]{\mute{\decorateTD\textcolor{red}{(TODO:~{#1})}}}

\colorlet{ftcolor}{Aquamarine}
\colorlet{fbcolor}{LimeGreen}
\colorlet{gdcolor}{RawSienna}
\colorlet{egcolor}{Plum}
\colorlet{ghcolor}{Orange}

\newcommand{\ft}[1]{\signedComment{FT}{ftcolor}{#1}}
\newcommand{\fb}[1]{\signedComment{FB}{fbcolor}{#1}}
\newcommand{\gd}[1]{\signedComment{GD}{gdcolor}{#1}}
\newcommand{\eg}[1]{\signedComment{EG}{egcolor}{#1}}
\newcommand{\gh}[1]{\signedComment{GH}{ghcolor}{#1}}

\newcommand{\cF}{\mathcal{F}}


\newcommand{\CF}{\mathcal{F}}
\newcommand{\CP}{\mathcal{P}}
\newcommand{\CT}{\mathcal{T}}

\newcommand{\systemstyle}[1]{\textsc{#1}}

\newcommand{\coc}{\systemstyle{Calculus of Constructions}}
\newcommand{\dedukti}{\systemstyle{Dedukti}}
\newcommand{\lc}{\systemstyle{\ensuremath{\lambda\text{-calculus}}}}
\newcommand{\lcs}{\systemstyle{\ensuremath{\lambda\text{-calculi}}}}
\newcommand{\lpc}{\systemstyle{\ensuremath{\lambda \Pi\text{-calculus}}}}
\newcommand{\lpcm}{\systemstyle{\ensuremath{\lambda \Pi\text{-calculus modulo theory}}}}
\newcommand{\lprolog}{\systemstyle{\ensuremath{\lambda\text{-Prolog}}}}



\def\Graph{{\mbox{\it Graph}}}
\def\Node{{\mbox{\it Node}}}
\def\omicron{{\mbox{\it omicron}}}
\def\arr{{\mbox{\it arrow}}}
\input{lambdapi.tex}

\begin{document}
\thispagestyle{empty}
\maketitle


\section*{Plan}

Introduction : papier précédent, explications de la structure des graphes pointés, apparition de Dedukti depuis (passage de Deduction modulo à Lambpdapi calcul modulo), théorie U (arrow, omicron) \\

The theory of pointed graphs : règles de réécritures \\

The language of pointed graphs : types, signature \\

Formulas : nécessité de restreindre l'ensemble des propositions dans la compréhension, définition des formules et des connecteurs sur les formules, principe de l'induction, principe de l'interprétation/valuation/update, présentations des nouveaux lemmes ajoutés pour prouver les résultats sur les formules (formulas-lemmas.lp) \\

Lemmas : expliquer qu'on n'a pas besoin des lemmes 1 et 2, explication des mécanismes classiques réutilisés dans chaque preuve, détails de preuves pour certains lemmes intéressants, extensionalité faible ajoutée entre 41 et 42 parce qu'elle est utilisée dans 44, 47, 48 \\

Conclusion : tableau récapitulatif du nombre de lignes de chaque preuve

\section{Introduction}

papier + dedukti + bout de théorie U (arrow, omicron)

\section{The language of pointed graphs}

\subsection{Sorts}

The langage of the theory IZmod uses four sorts. The first two are for
the pointed graphs and for the nodes of the pointed graphs.  In
\dedukti, we thus introduce two constants $\Graph$ and $\Node$ of type
$\Type$
\begin{lstlisting}
constant symbol Graph : TYPE;
constant symbol Node : TYPE;
\end{lstlisting}
As we have two sorts, we would need two universal quantifiers and two
existential quantifiers, one for each sort.  We rather use another
solution \cite{theoryU} that is to declare a constant $\Set$ of type
$\Type$ for codes of sorts, a function $\El$ of type $\Set \ra \Type$,
two constants $g$ and $n$ of sort $\Set$ and rewrite rules
$$\El~g \lra \Graph$$
$$\El~n \lra \Node$$

The two other sorts of the theory IZmod are for classes of nodes and
for binary relations on nodes.  In \dedukti, the sort of classes is
just $\Node \ra \Prop$ and that of binary relations
$\Node \ra \Node \ra \Prop$. To quantify on such sorts, we introduce
constants $\omicron$ of type $\Set$ and $\arr $ of type
$\Set \ra \Set \ra \Set$ and rewrite rules
$$\El~\omicron \lra \Prop$$
$$\El~(x~\arr~y) \lra (\El~x) \ra (\El~y)$$
Note that $\El~(n~\arr~\omicron)$ reduces to $\Node \ra \Prop$ and 
$\El~(n~\arr~(n~\arr~\omicron))$ reduces to $\Node \ra \Node \ra \Prop$. 

The symbols $\Graph$, $\Node$, $g$ and $n$ are specific to the
expression of $IZmod$ in \dedukti. In contrast the symbols $\Set$,
$\El$, $\omicron$, and $\arr$ are part of the standard library of 
\dedukti.

\subsection{Signature}

The signature of $IZmod$ contains 31 symbols. As we have replaced the
sorts for classes and relations with the \dedukti types
$\Node \ra \Prop$ and $\Node \ra \Node \ra \Prop$, we do not need
specific predicate symbols to apply a class to a node or a relations
to two. In the same way, we do not need comprehension axioms for 
classes and relations. 

In the same way, the equality symbol is part of the standard library of 
\dedukti.

So the signature is reduced to 26 symbols. 

\begin{lstlisting}
symbol eta : Graph → Node → Node → Prop;
symbol root : Graph → Node;
symbol cr : Graph → Node → Graph;
constant symbol o : Node;
constant symbol O : Node;
symbol i : Node → Node;
symbol i' : Node → Node;
symbol j : Node → Node;
symbol j' : Node → Node;
symbol I : Node → Prop;
symbol J : Node → Prop;
symbol ρ : Graph → Node;
symbol ρ^ : Node → Graph;
symbol Succ : Node → Node;
symbol Pred : Node → Node;
symbol Null : Node → Prop;
symbol Nat : Node → Prop;
symbol < : Node → Node → Prop;
symbol simeq : Graph → Graph → Prop;
symbol ∈ : Graph → Graph → Prop;
symbol join : Graph → Graph;
symbol pair : Graph → Graph → Graph;
symbol powerset : Graph → Graph;
symbol comp : Graph → El valuation → El formula → Graph;
symbol omega : Graph;
symbol Cl : Graph → Graph;
\end{lstlisting}


\section{The language of formulas}
\subsection{Formulas}

$$comp : \Graph \ra \El \ valuation \ra \El \ formula \ra \Graph$$

In $comp$ we may have noticed the presence of valuation and formula. 
Indeed, the lemmas in which the $comp$ symbol is used are only valid for a subset of formula. The formulas need to have all of its quantifiers of sort $Graph$ and can only use the language $\in$, $\simeq$ and the classic logical connectives.

Thus we need to introduce a set to caraterize the validity domain of such lemmas. In order to achieve this goal, we define the constant $formula$ of type $\Set$ and the logical connectives related to this constant.

\begin{lstlisting}
constant symbol formula : Set;
constant symbol eqF : El nat → El nat → El formula;
constant symbol inF : El nat → El nat → El formula;
constant symbol andF : El formula → El formula → El formula;
constant symbol orF : El formula → El formula → El formula;
constant symbol allF : El nat → El formula → El formula;
constant symbol exF : El nat → El formula → El formula;
constant symbol impF : El formula → El formula → El formula;
constant symbol fF : El formula;
constant symbol tF : El formula;
\end{lstlisting}

\subsection{Interpretation}

The next step is to interpret an object of type $formula$ into $Prop$. We introduce the constant $interpretation$ which receives a valuation of type $\El \ valuation$ and a formula of type $\El \ formula$ and return a $\Prop$.

\begin{lstlisting}
symbol interpretation : (El nat → Graph) → El formula → Prop;
\end{lstlisting}

The interpreation of formulas obeys the following rewriting rules :

$$interpretation \ \sigma \ (eqF \ x \ y) \lra (\sigma \ x) \ \simeq \ (\sigma \ y)$$
$$interpretation \ \sigma \ (inF \ x \ y) \lra (\sigma \ x) \ \in \ (\sigma \ y)$$
$$interpretation \ \sigma \ (andF \ f \ g) \lra (interpretation \ \sigma \ f) \conj (interpretation \ \sigma \ g)$$
$$interpretation \ \sigma \ (orF \ f \ g) \lra (interpretation \ \sigma \ f) \disj  (interpretation \sigma \ g)$$
$$interpretation \ \sigma \ (impF \ f \ g) \lra (interpretation \ \sigma \ f) \imp (interpretation \ \sigma \ g)$$
$$interpretation \ \sigma \ fF \lra \blbot$$
$$interpretation \ \sigma \ tF \lra \bltop$$

We need to have a tool to update a valuation when we assign a variable. To do so, we introduce the constant $update$ of type $(\El \ nat \ra Graph) \ra \El \ nat \ra Graph \ra (\El \ nat \ra Graph)$ which takes as arguments a valuation, a natural number and a graph and returns a new valuation. 

+ A finir avec update1
+ supprimer El valuation ou pas ?

\subsection{Theorems about valuation}

\begin{theorem}
$\fa \ \sigma, \fa \ x, y, z, \fa \ a, [(eqNP \ x \ y) \ \imp \ (((update1 \ \sigma \ x \ a) \ y \ z) \ \simeq \ a)] \ \conj \ [\blneg(eqNP \ x \ y) \ \imp \ (((update1 \ \sigma \ x \ a) \ y \ z) \ \simeq \ (\sigma \ z))]$
\end{theorem}

\begin{theorem}
$\fa \ \sigma, \fa \ x, \fa \ a, b, (a \ \simeq \ b) \ \imp \ [\fa y, (update \ \sigma \ x \ a \ y) \ \simeq \ (update \ \sigma \ x \ b \ y)]$
\end{theorem}

\begin{theorem}
$\fa \ \sigma, \fa \ x, y, \fa \ a, b, c, (a \ \simeq \ b) \ \imp \ (\fa z, (update \ (update \ \sigma \ x \ a) \ y \ c \ z) \ \simeq \ (update \  (update \ \sigma \ x \ b) \ y \ c \ z))$
\end{theorem}

\begin{theorem}
$\fa \ \sigma, \sigma', \fa \ x, \fa \ c, (\fa y, \sigma \ y \ \simeq \ \sigma' \ y) \ \imp \ (\fa z, (update \ \sigma \ x \ c \ z) \ \simeq \ (update \ \sigma' \ x \ c \ z))$
\end{theorem}

\begin{theorem}
$\fa \ f, \fa \ \sigma, \sigma', (interpretation \ \sigma \ f) \ \conj \ (\fa x, (\sigma \ x) \ \simeq \ (\sigma' \ x))) \ \imp \ (interpretation \ \sigma' \ f)$
\end{theorem}

\section{The theory of pointed graphs}


\section{Lemmas}


\section{Conclusion}


\section*{References}


\end{document}
