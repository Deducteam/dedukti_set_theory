\documentclass[a4paper]{article}

\usepackage[top=2.5cm, bottom=2.5cm, left=2.5cm, right=2.5cm]{geometry}
\usepackage{ae,lmodern}
\usepackage[francais]{babel}
\usepackage[utf8]{inputenc}
\usepackage[T1]{fontenc}

%\usepackage{paralist, tabularx}
\usepackage{oldlfont,amssymb,epsf,stmaryrd,epsfig,color}\usepackage{amsmath}
%\usepackage{hyperref}
%\usepackage{framed,multicol,changepage}
%\usepackage{graphicx}
\usepackage[dvipsnames]{xcolor}
%\usepackage{skull,ifoddpage,marginnote}
%\usepackage{pifont}
%\usepackage{graphicx}
%\usepackage{ebproof}
%\usepackage{mathtools}
%\usepackage{bm}
%\usepackage{stmaryrd}
%\usepackage{cleveref}
%\usepackage{tikz}
%\usepackage{import} % For inkscape generated files


\def\Type{\mbox{\tt TYPE}}
\def\Kind{\mbox{\tt KIND}}
\def\ra{\rightarrow}
\def\lra{\longrightarrow}
\def\Set{\mbox{\it Set}}
\def\bliota{{\color{blue} ind}}
\def\El{{\mbox{\it El}}}
\def\Prop{{\mbox{\it Prop}}}
\def\Prf{{\color{blue} \mbox{\it Prf}}}
\def\imp{\mathbin{\color{blue} \Rightarrow}}
\def\fa{{\color{blue} \forall}}
\def\bltop{{\color{blue} \top}}
\def\blbot{{\color{blue} \bot}}
\DeclareMathOperator{\blneg}{{\color{blue} \neg}}
\def\conj{\mathbin{\color{blue} \wedge}}
\def\disj{\mathbin{\color{blue} \vee}}
\def\ex{{\color{blue} \exists}}
\def\S{{\color{blue} \mbox{\it succ}}}
\def\0{{\color{blue} 0}}
\def\positive{{\color{blue} \mbox{\it positive}}}
\def\pred{{\color{blue} \mbox{\it pred}}}
\def\Prfc{{\color{blue} \mbox{\it Prf}_c}}
\def\impc{\mathbin{\color{blue} \Rightarrow_c}}
\def\conjc{\mathbin{\color{blue} \wedge_c}}
\def\disjc{\mathbin{\color{blue} \vee_c}}
\def\fac{{\color{blue} \forall_c}}
\def\exc{{\color{blue} \exists_c}}
\def\o{{\color{blue} o}}
\def\arr{\mathbin{\color{blue} \leadsto}}%\mbox{\it arrow}}}
\def\arrd{\mathbin{\color{blue} \leadsto_d}}%\mbox{\it arrow}_d}}
\def\impd{\mathbin{\color{blue} \Rightarrow_d}}
\def\blpi{{\color{blue} \pi}}
\def\Scheme{{\color{blue} \mbox{\it Scheme}}}
\def\Els{{\color{blue} \mbox{\it Els}}}
\def\lift{{\color{blue} \uparrow}}
\def\calfa{{\rotatebox[origin=c]{180}{{\color{blue} A}}}}
\def\sffa{{\rotatebox[origin=c]{180}{{\color{blue}\ensuremath{\mathcal{A}}}}}}
\def\pos{\mbox{\it positive}}
\def\psub{{\color{blue} \mbox{\it psub}}}
\def\pair{{\color{blue} \mbox{\it pair}}}
\def\fst{{\color{blue} \mbox{\it fst}}}
\def\snd{{\color{blue} \mbox{\it snd}}}
\def\eq{{\mbox{\it eq}}}
\def\refl{{\mbox{\it refl}}}

\newtheorem{definition}{Definition}[section]
\newtheorem{theorem}{Theorem}[section]
\newtheorem{lemma}{Lemma}[section]
\newtheorem{corollary}{Corollary}[section]
\newtheorem{property}{Property}[section]

\newenvironment{proof}{\noindent {\em Proof.}}{\medskip}
\newenvironment{example}{\noindent {\em Example.}}{\medskip}
\newenvironment{remark}{\noindent {\em Remark.}}{\medskip}


\newcommand{\coc}{\textsc{Calculus of Constructions}}
\newcommand{\dedukti}{\textsc{Dedukti}}
\newcommand{\lc}{$\lambda\text{-calculus}$}
\newcommand{\lcs}{$\lambda\textit{-calculi}$}
\newcommand{\lpc}{$\lambda \Pi\textit{-calculus}$}
\newcommand{\lpcm}{$\lambda \Pi\textit{-calculus modulo theory}$}
\newcommand{\lprolog}{$\lambda\textit{-Prolog}$}

\def\Graph{{\mbox{\it Graph}}}
\def\Node{{\mbox{\it Node}}}
\def\omicron{{\mbox{\it omicron}}}
\def\arr{{\mbox{\it arrow}}}

\begin{document}

\section{Introduction}

papier + dedukti + bout de théorie U (arrow, omicron)

\section{The language of pointed graphs}

\subsection{Sorts}

The langage of the theory IZmod uses four sorts. The two first ones
are for the pointed graphs and for the nodes of the pointed graphs.

In Dedukti, we thus introduce two constants $\Graph$ and $\Node$ of
type $\Type$
\begin{verbatim}
constant symbol Graph : TYPE;
constant symbol Node : TYPE;
\end{verbatim}
As we have two sorts, we would need two universal quantifiers and two 
existential quantifiers, one on each sort. 
We rather use another solution \cite{theoryU} that is to declare a constant
$\Set$ of type $\Type$ for codes of sorts,
a function $\El$ of type $\Set \ra \Type$, two 
$g$ and $n$ of sort $\Set$ and rewrite rules 
$$\El~g \lra \Graph$$
$$\El~n \lra \Node$$

The two other sorts of the theory IZmod are for classes of nodes and
for binary relations on nodes.

In Dedukti, the sort of classes is just $\Node \ra \Prop$ and that of
binary relations $\Node \ra \Node \ra \Prop$. To be able to quantify on 
such sorts, we introduce constants $\omicron$ of type $\Set$ 
$\arr $ of type
$\Set \ra \Set \ra \Set$ 
and rewrite rules
$$El~\omicron \lra \Prop$$
$$(El~(x \arr y)) \lra (El x) \ra (El y)$$

Note that $El (n \arr \omicron) \lra \Node \ra \Prop$ and 
$El (n \arr (n \arr \omicron)) \lra \Node \ra \Node \ra \Prop$. 



\section{Formulas}

\section{Lemmas}

Expliquer qu'on n'a pas besoin des lemmes 1 et 2
Extensionnalité faible ajoutée entre 41 et 42 parce qu'elle est utilisée dans 44, 47, 48 



\end{document}